\documentclass{scrartcl}

% Compiled with assistance from a wonderful guide by Brett Witty: https://www.brettwitty.net/how-to-design-cards-1.html; https://www.brettwitty.net/how-to-design-cards-2.html; https://www.brettwitty.net/how-to-design-cards-3.html

\usepackage{pdflscape}
\usepackage{xcolor}
\usepackage{tikz}
\usetikzlibrary{positioning}

\usepackage{pifont}
\usepackage{graphicx}
\usepackage{etoolbox}
\usepackage[utf8]{inputenc}

\usepackage{multicol}
\usepackage[margin=8mm]{geometry}

\usepackage{nopageno}
\usepackage{graphicx, txfonts}
\usepackage{anyfontsize}
\usepackage{ifthen}
\usepackage{array}

\RequirePackage[active,tightpage]{preview}
\PreviewEnvironment{tikzpicture}
\setlength{\PreviewBorder}{0.125in}

\hyphenpenalty=100000

% All cards
\newcommand{\cardmargin}{2mm}
\newcommand{\cardcorners}{4mm}
\newcommand{\cardwidth}{63mm}
\newcommand{\cardheight}{88mm}
\newcommand{\cardnameheight}{10mm}
\newcommand{\carddescriptionheight}{30mm}
\newcommand{\carddescriptionmargin}{4mm}

% TODO apply internationalization rules here too
\newcommand{\card}[1]{\textbf{#1}}
\newcommand{\act}{\textbf{Action}}
\newcommand{\stn}{\textbf{Stance}}
\newcommand{\fin}{\textbf{Final}}
\newcommand{\mis}{\textbf{Mission}}
%\newcommand{\brg}{\textcolor{white}\bargainheart{\textcolor{black}\bargainheartoutline}}
\newcommand{\brg}{{\ooalign{\textcolor{white}\bargainheart\cr\hss\textcolor{black}\bargainheartoutline\kern .025em\hss}}}
\newcommand{\btr}{\betrayalheart}


\newcommand{\brgbrg}{\brg\!\!\brg}
\newcommand{\btrbtr}{\btr\!\!\btr}
\newcommand{\btrbrg}{\btr\!\!\brg}
\newcommand{\brgbtr}{\brg\!\!\btr}

% Stance Maps
\newcommand{\matmargins}{30mm}
\newcommand{\matwidth}{267mm}
\newcommand{\matheight}{180mm}
\newcommand{\matpadding}{3mm}

\newcommand{\matbaseangle}{261}
\newcommand{\matrotationper}{33}

\newcommand{\matcardexperimental}[2]{
    \draw[rounded corners = \cardcorners, #1, rotate around={\matbaseangle + #2 * \matrotationper:(0.5 * \matwidth, 0.8 * \cardwidth + \cardmargin)}] (0.5 * \matwidth - 0.5 * \cardwidth, 1.4 * \cardwidth + \cardmargin) rectangle ++(\cardwidth - \cardmargin, \cardheight - \cardmargin);
}

\newcommand{\stancematexperimental}[1]{ % Name, Description, Instant?, Persistent?
    \begin{tikzpicture}

        \draw[rounded corners = \cardcorners] (0,0) rectangle (\matwidth, \matheight);

        \newcounter{Rotations}

        \matcard{red}{0}
        \matcard{red}{1}
        \matcard{red}{2}
        \matcard{red}{3}
        \matcard{red}{4}
        \matcard{red}{5}
        \matcard{red}{6}
    
        \node (name) at (0.5 * \cardwidth - 0.5 * \cardmargin, \carddescriptionheight + \cardnameheight) [text width = \cardwidth - 3 * \cardmargin, minimum height = \cardnameheight, align = center] {\Large};
    
    \end{tikzpicture}
}

\newcommand{\matinner}{1mm}

\newcommand{\matcard}[4]{

    \ifthenelse{\equal{#3}{#4}}
        {
            \node (pname) at (\cardwidth * #1 + \matinner * #1 + \cardmargin + \cardwidth * 0.5, \matpadding + \cardheight * #2 + \cardheight * 0.9) [align = center] {YOU (NOTES \& POINTS)};
        }
        {
            \draw[rounded corners = \cardcorners, draw] (\matpadding + \cardwidth * #1 + \matinner * #1, \matpadding + \cardheight * #2) rectangle ++(\cardwidth - \cardmargin, \cardheight - \cardmargin);
            \node (pname) at (\cardwidth * #1 + \matinner * #1 + \cardmargin + \cardwidth * 0.5, \matpadding + \cardheight * #2 + \cardheight * 0.5) [align = center] {PLAYER #3};
        }
}

\newcommand{\stancemat}[1]{ % Player #
    \begin{tikzpicture}

        \draw[white] (0,0) rectangle (\matwidth, \matheight);

        \matcard{0}{0}{1}{#1}
        \matcard{1}{0}{2}{#1}
        \matcard{2}{0}{3}{#1}
        \matcard{3}{0}{4}{#1}

        \matcard{0}{1}{4}{#1}
        \matcard{1}{1}{5}{#1}
        \matcard{2}{1}{6}{#1}
        \matcard{3}{1}{7}{#1}
    
    \end{tikzpicture}
}

% Actions
\newcommand{\action}[4]{ % Name, Description, Instant?, Persistent?
    \begin{tikzpicture}

        \draw[rounded corners = \cardcorners] (0,0) rectangle (\cardwidth - \cardmargin, \cardheight - \cardmargin);
    
        \node (name) at (0.5 * \cardwidth - 0.5 * \cardmargin, \carddescriptionheight + \cardnameheight) [text width = \cardwidth - 3 * \cardmargin, minimum height = \cardnameheight, align = center] {\Large #1};
    
        \node (description) at (0.5 * \cardwidth - 0.5 * \cardmargin, 0.5 * \carddescriptionheight + \carddescriptionmargin) [draw = lightgray, fill = white, thick, text width = \cardwidth - 3 * \carddescriptionmargin, minimum height = \carddescriptionheight, align = center] {
            \ifstrempty{#3}{}{
                % Instant, add the instant icon
                \small \textbf{Instant} \textit{#3}

            }\ifstrempty{#4}{}{
                % Persistent, add the instant icon
                \small \textbf{Persistent}.

            }
            \small #2
        };
    
    \end{tikzpicture}
}

% Stances
\newcommand{\bargainheart}{\ensuremath\varheartsuit}
\newcommand{\bargainheartoutline}{\ensuremath\heartsuit}
\newcommand{\betrayalheart}{\rotatebox[origin=c]{180}{\ensuremath\varheartsuit}}

\newcommand{\heart}[1]{
    \ifthenelse{\equal{#1}{BARGAIN}}
        {\bargainheartoutline}
        {\betrayalheart}
}

\newcommand{\stance}[5]{ % Name, Description, Type, AgainstBargain, AgainstBetrayal
    \begin{tikzpicture}

        \draw[rounded corners = \cardcorners] (0,0) rectangle (\cardwidth - \cardmargin, \cardheight - \cardmargin);
    
        \node (type) at (0.12 * \cardwidth + \cardmargin, 0.88 * \cardheight) [align = left] {\fontsize{50px}{50px} \heart{#3}};

        \node (basevsbargain) at (0.75 * \cardwidth, 0.88 * \cardheight + 1.1mm) [rounded corners = \cardcorners, align = center] {\textcolor{white}{\fontsize{50px}{60px} \bargainheart{}}};
        \node (basevsbargainoutline) at (0.75 * \cardwidth, 0.88 * \cardheight + 1.1mm) [rounded corners = \cardcorners, align = center] {\fontsize{50px}{60px} \bargainheartoutline{}};
        \node (basevsbetrayal) at (0.86 * \cardwidth, 0.88 * \cardheight - 1.1mm) [rounded corners = \cardcorners, align = center] {\fontsize{50px}{60px} \betrayalheart{}};

        \node (basevsbargaintext) at (0.75 * \cardwidth + 0.5mm, 0.88 * \cardheight + 1.5mm) [rounded corners = \cardcorners, align = center] {#4};
        \node (basevsbetrayaltext) at (0.86 * \cardwidth + 0.5mm, 0.875 * \cardheight - 1mm) [rounded corners = \cardcorners, align = center] {\textcolor{white}{#5}};

        \ifstrempty{#2}
            {
                \node (name) at (0.5 * \cardwidth - 0.5 * \cardmargin, 0.5 * \cardnameheight) [text width = \cardwidth - 3 * \cardmargin, minimum height = \cardnameheight, align = center] {\Large #1};
            }
            {
                \node (name) at (0.5 * \cardwidth - 0.5 * \cardmargin, \carddescriptionheight + \cardnameheight) [text width = \cardwidth - 3 * \cardmargin, minimum height = \cardnameheight, align = center] {\Large #1};
                \node (description) at (0.5 * \cardwidth - 0.5 * \cardmargin, 0.5 * \carddescriptionheight + \carddescriptionmargin) [draw = lightgray, fill = white, thick, text width = \cardwidth - 3 * \carddescriptionmargin, minimum height = \carddescriptionheight, align = center] {\small #2};
            }
        
    \end{tikzpicture}
}

%TODO Finals
\newcommand{\final}[3]{ % Name, Description, Priority
\begin{tikzpicture}

    \draw[rounded corners = \cardcorners] (0,0) rectangle (\cardwidth - \cardmargin, \cardheight - \cardmargin);

    \node (name) at (0.5 * \cardwidth - 0.5 * \cardmargin, \carddescriptionheight + \cardnameheight) [text width = \cardwidth - 3 * \cardmargin, minimum height = \cardnameheight, align = center] {\Large #1};
    
    \node (priority) at (0.5 * \cardwidth - 0.5 * \cardmargin, \cardheight - \cardmargin - 0.5 * \cardnameheight) [minimum height = \cardnameheight, align = right] {\LARGE #3};

    \node (description) at (0.5 * \cardwidth - 0.5 * \cardmargin, 0.5 * \carddescriptionheight + \carddescriptionmargin) [draw = lightgray, fill = white, thick, text width = \cardwidth - 3 * \carddescriptionmargin, minimum height = \carddescriptionheight, align = center] {
        \small #2
    };

\end{tikzpicture}
}

%TODO Missions

%TODO images

\begin{document}

\thispagestyle{empty}

\stancemat{1}
\stancemat{2}
\stancemat{3}
\stancemat{4}
\stancemat{5}
\stancemat{6}
\stancemat{7}
\stancemat{8}

% RULES

\begin{tikzpicture}

    \draw[rounded corners = \cardcorners] (0,0) rectangle (\cardwidth - \cardmargin, \cardheight - \cardmargin);
    
    \node (title) at (0.5 * \cardwidth - 0.5 * \cardmargin, \cardheight - \cardmargin - 0.5 * \cardnameheight) [minimum height = \cardnameheight] {\LARGE CONSPIRACY};
    \node (subtitle) at (0.5 * \cardwidth - 0.5 * \cardmargin, \cardheight - \cardmargin - 1 * \cardnameheight) [minimum height = 0.5 * \cardnameheight] {RULES 1/3};
    \node (phase) at (0.5 * \cardwidth - 0.5 * \cardmargin, \cardheight - \cardmargin - 1.4 * \cardnameheight) [minimum height = 0.5 * \cardnameheight] {\small \textit{PREGAME}};

    \node (rules) at (0.5 * \cardwidth - 0.5 * \cardmargin, \cardheight - \cardmargin - 5 * \cardnameheight) [text width = \cardwidth - 3 * \cardmargin, minimum height = 0.5 * \cardnameheight, align = left] { 
        \footnotesize 
        For \textit{n} players, deal each player:\\
        ~~\textit{n} + 2 \stn{} cards\\
        ~~4 \act{} cards\\
        ~~4 \fin{} cards \\
        ~\\
        Each player should prepare a \stn{} mat, which should be a page with a space for every other player. During the game, \stn{} cards will be placed here to indicate your bargains and betrayals. \\
        ~\\
        Remaining \act{} cards should be placed in a deck, face down, in the center of the table. This is the \textbf{draw deck}. \\
        ~\\
        The first player is the person whose birthday is closest to July 6th. \\
    };

\end{tikzpicture}

\begin{tikzpicture}

    \draw[rounded corners = \cardcorners] (0,0) rectangle (\cardwidth - \cardmargin, \cardheight - \cardmargin);
    
    \node (title) at (0.5 * \cardwidth - 0.5 * \cardmargin, \cardheight - \cardmargin - 0.5 * \cardnameheight) [minimum height = \cardnameheight] {\LARGE CONSPIRACY};
    \node (subtitle) at (0.5 * \cardwidth - 0.5 * \cardmargin, \cardheight - \cardmargin - 1 * \cardnameheight) [minimum height = 0.5 * \cardnameheight] {RULES 2/3};
    \node (phase) at (0.5 * \cardwidth - 0.5 * \cardmargin, \cardheight - \cardmargin - 1.4 * \cardnameheight) [minimum height = 0.5 * \cardnameheight] {\small \textit{ROUNDS}};

    \node (rules) at (0.5 * \cardwidth - 0.5 * \cardmargin, \cardheight - \cardmargin - 5 * \cardnameheight) [text width = \cardwidth - 3 * \cardmargin, minimum height = 0.5 * \cardnameheight, align = left] { 
        \footnotesize 
        Each player must do the following in their turn: \\
        1. If the player has played a \stn{} against every other player, their turn is skipped. Otherwise: \\
        2. Play a \stn{} face-down, unless otherwise specified. \\
        \footnotesize 3. Optionally play one \act{}, and resolve its effect. \\
        4. If you are holding fewer than 8 \act{} cards, draw an \act{} from the draw deck. \\
        5. Your turn is now over, and the player to your left begins! \\
        ~\\
        This continues until every player has filled their \stn{} mat. \\
    };

\end{tikzpicture}

\begin{tikzpicture}

    \draw[rounded corners = \cardcorners] (0,0) rectangle (\cardwidth - \cardmargin, \cardheight - \cardmargin);
    
    \node (title) at (0.5 * \cardwidth - 0.5 * \cardmargin, \cardheight - \cardmargin - 0.5 * \cardnameheight) [minimum height = \cardnameheight] {\LARGE CONSPIRACY};
    \node (subtitle) at (0.5 * \cardwidth - 0.5 * \cardmargin, \cardheight - \cardmargin - 1 * \cardnameheight) [minimum height = 0.5 * \cardnameheight] {RULES 3/3};
    \node (phase) at (0.5 * \cardwidth - 0.5 * \cardmargin, \cardheight - \cardmargin - 1.4 * \cardnameheight) [minimum height = 0.5 * \cardnameheight] {\small \textit{POST-ROUND}};

    \node (rules) at (0.5 * \cardwidth - 0.5 * \cardmargin, \cardheight - \cardmargin - 5 * \cardnameheight) [text width = \cardwidth - 3 * \cardmargin, minimum height = 0.5 * \cardnameheight, align = left] { 
        \footnotesize 
        Once all \stn{} mats have been filled, the ??? part of the round is complete. Before revealing each of your \stn{} cards, each player may choose and play a \fin{} card, face-down. \\
        ~\\
        At the same time, all players must flip their played \stn{} cards face-up, along with their played \fin{} card. \\
        ~\\
        Resolve the effects of \fin{} cards in order of priority, lowest-to-highest. Once they have all been resolved, calculate points by resolving all \stn{} interactions in a clockwise direction. Then, go onto the next round!
        Play 4 rounds if there are fewer than 5 players, otherwise play 3 rounds. \\
    };

\end{tikzpicture}

% QUICK REFERENCE

% ACTIONS
\action{NIL}{All players must immediately play a \stn{} card.}{}{}
\action{CURIOSITY}{All players holding at least one \btr{} must raise their hands. You may swap any held \stn{} of yours for a \btr{} of their choice.}{}{}
\action{CACOPHONY}{All players must immediately reveal their \textbf{Cacophony} cards. If there are 3 or fewer other \textbf{Cacophony} cards revealed, gain that many points. Otherwise, discard all your \act{} cards.}{}{}
\action{CACOPHONY}{All players must immediately reveal their \textbf{Cacophony} cards. If there are 3 or fewer other \textbf{Cacophony} cards revealed, gain that many points. Otherwise, discard all your \act{} cards.}{}{}
\action{CACOPHONY}{All players must immediately reveal their \textbf{Cacophony} cards. If there are 3 or fewer other \textbf{Cacophony} cards revealed, gain that many points. Otherwise, discard all your \act{} cards.}{}{}
\action{CACOPHONY}{All players must immediately reveal their \textbf{Cacophony} cards. If there are 3 or fewer other \textbf{Cacophony} cards revealed, gain that many points. Otherwise, discard all your \act{} cards.}{}{}
\action{CACOPHONY}{All players must immediately reveal their \textbf{Cacophony} cards. If there are 3 or fewer other \textbf{Cacophony} cards revealed, gain that many points. Otherwise, discard all your \act{} cards.}{}{}
\action{MISERY}{All other players discard 2 \act{} cards. If \textbf{Joy} is in play, double this effect.}{}{true}
\action{JOY}{You may play 3 \act{} cards immediately. If \textbf{Misery} has been played this round, you may play 3 \act{} cards immediately instead.}{}{true}
\action{IMPLOSION}{Cancel the effects of another player's \act{}.}{At any point while a player is playing an \act{}.}{}
\action{THAT WHICH is UNSEEN}{You may look at the \act{} draw deck. You may order it or shuffle it in any way you wish before putting it back.}{}{}
\action{PARANOIA}{All players holding at least one \brg{} must raise their hands. You may swap any held \stn{} of yours with a \brg{} of their choice.}{}{}
\action{DUPLICITY}{You may swap two of your played \stn{} cards.}{}{}
\action{BURNING TRUTH}{Choose any player who must reveal their held \stn{} cards to everyone, then give that player this \textbf{Burning Truth} card instead of discarding it.}{}{}
\action{LANDMINE}{Negate that effect. The player must also discard all their \act{} cards.}{If a player tries to take or swap any of your held cards.}{}
\action{TOXIC SHOCK}{All other players must discard 2 \act{} cards.}{}{}
\action{SUBTERFUGE}{You may look at the played \stn{} cards of any one other player.}{}{}
\action{FAMINE}{While this card is in effect, no \act{} cards may be drawn.}{}{true}
\action{ABUNDANCE}{Remove \textbf{Famine} from play. Draw as many \act{} cards as there are players, then give each player one card of your choosing.}{}{}
\action{AMNESIA}{Remove all persistent \act{} cards from play.}{}{}
\action{HONESTY}{You may turn any of your own played \stn{} cards face-up.}{}{}
\action{FEAR}{While you hold this card, you may not play any other \act{} cards.}{}{}
\action{LOVE}{If another player has \textbf{Love}, they may reveal it and discard it, and you both gain 2 points. Only the first player to reveal will gain points. If no player reveals \textbf{Love}, you lose 2 points.}{}{}
\action{LOVE}{If another player has \textbf{Love}, they may reveal it and discard it, and you both gain 2 points. Only the first player to reveal will gain points. If no player reveals \textbf{Love}, you lose 2 points.}{}{}
\action{LOVE}{If another player has \textbf{Love}, they may reveal it and discard it, and you both gain 2 points. Only the first player to reveal will gain points. If no player reveals \textbf{Love}, you lose 2 points.}{}{}
\action{TITHES}{All players must give you one of their \act{} cards, of their choosing.}{}{}
\action{PURIFICATION}{All players, including yourself, must reveal one of their played \stn{} cards and then return it to their hand.}{}{}
\action{TURNABOUT is FAIR PLAY}{You may pick the card up and play it yourself immediately afterwards.}{If an \act{} allows a player to take cards from your hand}{}

% STANCES
\stance{BARGAIN}{}{BARGAIN}{3}{-1}
\stance{BARGAIN}{}{BARGAIN}{3}{-1}
\stance{BARGAIN}{}{BARGAIN}{3}{-1}
\stance{BARGAIN}{}{BARGAIN}{3}{-1}
\stance{BARGAIN}{}{BARGAIN}{3}{-1}
\stance{BARGAIN}{}{BARGAIN}{3}{-1}
\stance{BARGAIN}{}{BARGAIN}{3}{-1}
\stance{BARGAIN}{}{BARGAIN}{3}{-1}
\stance{BARGAIN}{}{BARGAIN}{3}{-1}
\stance{BARGAIN}{}{BARGAIN}{3}{-1}
\stance{BARGAIN}{}{BARGAIN}{3}{-1}
\stance{BARGAIN}{}{BARGAIN}{3}{-1}
\stance{BARGAIN}{}{BARGAIN}{3}{-1}
\stance{BARGAIN}{}{BARGAIN}{3}{-1}
\stance{BARGAIN}{}{BARGAIN}{3}{-1}
\stance{BARGAIN}{}{BARGAIN}{3}{-1}
\stance{BARGAIN}{}{BARGAIN}{3}{-1}
\stance{BARGAIN}{}{BARGAIN}{3}{-1}
\stance{BARGAIN}{}{BARGAIN}{3}{-1}
\stance{BARGAIN}{}{BARGAIN}{3}{-1}
\stance{BARGAIN}{}{BARGAIN}{3}{-1}
\stance{BARGAIN}{}{BARGAIN}{3}{-1}
\stance{BARGAIN}{}{BARGAIN}{3}{-1}
\stance{BARGAIN}{}{BARGAIN}{3}{-1}
\stance{BARGAIN}{}{BARGAIN}{3}{-1}
\stance{BETRAYAL}{}{BETRAYAL}{2}{0}
\stance{BETRAYAL}{}{BETRAYAL}{2}{0}
\stance{BETRAYAL}{}{BETRAYAL}{2}{0}
\stance{BETRAYAL}{}{BETRAYAL}{2}{0}
\stance{BETRAYAL}{}{BETRAYAL}{2}{0}
\stance{BETRAYAL}{}{BETRAYAL}{2}{0}
\stance{BETRAYAL}{}{BETRAYAL}{2}{0}
\stance{BETRAYAL}{}{BETRAYAL}{2}{0}
\stance{BETRAYAL}{}{BETRAYAL}{2}{0}
\stance{BETRAYAL}{}{BETRAYAL}{2}{0}
\stance{BETRAYAL}{}{BETRAYAL}{2}{0}
\stance{BETRAYAL}{}{BETRAYAL}{2}{0}
\stance{BETRAYAL}{}{BETRAYAL}{2}{0}
\stance{BETRAYAL}{}{BETRAYAL}{2}{0}
\stance{BETRAYAL}{}{BETRAYAL}{2}{0}
\stance{BETRAYAL}{}{BETRAYAL}{2}{0}
\stance{BETRAYAL}{}{BETRAYAL}{2}{0}
\stance{BETRAYAL}{}{BETRAYAL}{2}{0}
\stance{BETRAYAL}{}{BETRAYAL}{2}{0}
\stance{BETRAYAL}{}{BETRAYAL}{2}{0}
\stance{BETRAYAL}{}{BETRAYAL}{2}{0}
\stance{BETRAYAL}{}{BETRAYAL}{2}{0}
\stance{BETRAYAL}{}{BETRAYAL}{2}{0}
\stance{BETRAYAL}{}{BETRAYAL}{2}{0}
\stance{BETRAYAL}{}{BETRAYAL}{2}{0}
\stance{SYMBIOSIS}{\brg{}: both players gain an additional 2 points.}{BARGAIN}{3}{-1}
\stance{EXCOMMUNICATION}{This card must be played face-up.}{BETRAYAL}{4}{0}
\stance{QUID PRO QUO}{At any time once played, this card can be flipped face-up to negate a player's \act{}.}{BARGAIN}{3}{-1}
\stance{ASSASSINATION}{\brg{}: the other player loses 2 additional points.}{BETRAYAL}{1}{0}
\stance{FAIR DEAL}{This card must be played face-up.}{BARGAIN}{4}{-1}
\stance{COUP d'ETAT}{If played against the player with the most points in the last round, they lose 2 additional points.}{BETRAYAL}{2}{0}
\stance{CONTRACTUAL OBLIGATIONS}{\btr{}: the other player loses 3 points.}{BARGAIN}{1}{1}
\stance{A TRUE NAME, SPOKEN SOFTLY}{}{BETRAYAL}{1}{1}
\stance{PEACEKEEPING}{\btr{}: you may take all \act{} and \fin{} cards of the other player.}{BARGAIN}{3}{-1}
\stance{“PEACEKEEPING”}{\brg{}: take all the \act{} cards of the other player.}{BETRAYAL}{2}{0}
\stance{CAMERADERIE}{}{BARGAIN}{4}{-2}
\stance{BITTER TASTE}{The other player must also redraw all their \brg{} cards at the beginning of the next round.}{BETRAYAL}{2}{0}
\stance{LOVE-TACKLE}{If a player reveals this card during the game, it is flipped face-up, and they must give you all their \act{} cards.}{BARGAIN}{3}{-1}
\stance{THROWN to the LAMPREYS}{This card must be played face-up and cannot be moved or interacted with. The player this card is played against must immediately play 2 of their \stn{} cards.}{BETRAYAL}{0}{0}
\stance{QUISLING}{\brg{}: both players gain an additional 2 points.}{BETRAYAL}{2}{0}
\stance{TRUE LOVE}{If this card is \brgbrg{} with another \textbf{True Love}, you do not lose any base points from \brgbtr{}.}{BARGAIN}{3}{-3}
\stance{TRUE LOVE}{If this card is \brgbrg{} with another \textbf{True Love}, you do not lose any base points from \brgbtr{}.}{BARGAIN}{3}{-3}
\stance{ZERO SUM}{Completely negate the points gained or lost by both players in this interaction.}{BETRAYAL}{0}{0}
\stance{SONNET}{If this is played \brgbtr{}, you may sing a song to avoid losing points.}{BARGAIN}{3}{-1}
\stance{HOSTILE TAKEOVER}{Gain 1 point for every \brg{} you have played.}{BETRAYAL}{-1}{-1}
\stance{the SUBLIME}{}{BARGAIN}{2}{0}
\stance{NIHILISM}{}{BETRAYAL}{2}{0}

% FINALS
\final{ENTROPY}{Cancel the effects of all \fin{} cards this round.}{0}
\final{CURSED DECK}{Swap two played \stn{} cards between any two players other than yourself.}{1}
\final{STOIC HYMNAL}{Your \stn{} cards cannot be changed by any future \fin{} effects this round.}{2}
\final{VOW of VENGEANCE}{If you have played a \brgbtr{}, you may take all their held \act{} and \fin{} cards.}{3}
\final{ACT of PENANCE}{If you have played a \btrbrg{}, you may change your stance to a \brg{}, if you are holding one. If you do so, the other player gains 2 extra points.}{4}
\final{LORD of the PIT}{Gain 3 points for every time you have played a \btrbtr{} in this round.}{5}
\final{PIETY}{Gain 3 points for each of your \brgbtr{} in this round.}{6}
\final{PERJURY}{Replace one of your played \stn{} cards with one of the opposite \stn{}, if you hold one.}{7}
\final{MARTYRDOM}{Choose another player. Both of you lose 5 points. If you are already the player with the lowest score before losing these points, only the other player loses 5 points.}{8}
\final{PANTHEON}{If more than half the \fin{} cards have already been resolved, gain 3 points. Otherwise, lose 1 point.}{9}
\final{the ABYSS}{The player of the next \fin{} to resolve loses 4 points.}{10}
\final{FLAGELLATION}{Lose 3 points. You may not lose any further points this round, for any reason.}{11}
\final{CHAOS}{Pick up 4 played \stn{} cards, shuffle them, and randomly deal them back to where they were taken from before revealing them.}{12}
\final{FUNERAL PYRE}{Take 3 points from one player, and give 3 points to a player other than yourself.}{13}
\final{INTERREGNUM}{Choose another player. All other players vote for you, or for the other player. If you win a majority of votes, take 3 points from the other player. If you do not, give 3 points to them.}{14}
\final{P'Zea-ia-Gwlfth}{If this is the last \fin{} to resolve, gain 5 points.}{15}
\final{ANARCHY}{Every player gains 1 point for every \brg{} they have played. You gain an extra point for each \brg{} you have played.}{16}
\final{BURNT OFFERINGS}{You may ignore any points lost due to any one \btr{}.}{17}
\final{MÓRRÍGAN}{Gain 1 point for every \brgbtr{} in this round.}{18}
\final{SACRIFICE}{Gain 3 points and discard all of your \act{} and \fin{} cards.}{19}
\final{CONTRITION}{Ask a player you have played a \btr{} against if they forgive you. If they do, you gain 3 points, and they gain 1 point.}{20}
\final{FINAL RITES}{Select 3 pairs of \brgbtr{} and swap them. You may only select 1 interaction involving yourself.}{100}

\end{document}